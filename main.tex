\documentclass[a4paper]{tufte-book}

\hypersetup{colorlinks}% uncomment this line if you prefer colored hyperlinks (e.g., for onscreen viewing)

%%
% Book metadata
\title{Design of Fault Detection, Isolation and Recovery in the AcubeSAT nanosatellite\thanks{AcubeSAT}}
\author[kongr45gpen]{kongr45gpen}
\publisher{Draft SpaceDot}

%%
% If they're installed, use Bergamo and Chantilly from www.fontsite.com.
% They're clones of Bembo and Gill Sans, respectively.
%\IfFileExists{bergamo.sty}{\usepackage[osf]{bergamo}}{}% Bembo
%\IfFileExists{chantill.sty}{\usepackage{chantill}}{}% Gill Sans

%\usepackage{microtype}

%%
% Just some sample text
\usepackage{lipsum}

%%
% For nicely typeset tabular material
\usepackage{booktabs}

%%
% For graphics / images
\usepackage{graphicx}
\setkeys{Gin}{width=\linewidth,totalheight=\textheight,keepaspectratio}
\graphicspath{{graphics/}}

% What
\setcounter{secnumdepth}{3}


%%
% Prints a trailing space in a smart way.
\usepackage{xspace}

% Inserts a blank page
\newcommand{\blankpage}{\newpage\hbox{}\thispagestyle{empty}\newpage}

\usepackage{hyperref}
\usepackage{cleveref}

\usepackage{siunitx}

\usepackage{acro}

\DeclareAcronym{FDIR}{short=FDIR,long={Fault Detection, Isolation and Recovery}}

% Generates the index
\usepackage{makeidx}
\makeindex

\begin{document}

% Front matter
%\frontmatter

% r.1 blank page
%\blankpage

% r.3 full title page
\maketitle


% v.4 copyright page
\newpage
\begin{fullwidth}
~\vfill
\thispagestyle{empty}
\setlength{\parindent}{0pt}
\setlength{\parskip}{\baselineskip}
Copyright \copyright\ \the\year\ \thanklessauthor

\par\smallcaps{Published by \thanklesspublisher}

\par\smallcaps{tufte-latex.github.io/tufte-latex/}

\par Licensed under the Apache License, Version 2.0 (the ``License''); you may not
use this file except in compliance with the License. You may obtain a copy
of the License at \url{http://www.apache.org/licenses/LICENSE-2.0}. Unless
required by applicable law or agreed to in writing, software distributed
under the License is distributed on an \smallcaps{``AS IS'' BASIS, WITHOUT
WARRANTIES OR CONDITIONS OF ANY KIND}, either express or implied. See the
License for the specific language governing permissions and limitations
under the License.\index{license}

\par\textit{First printing, 2021}
\end{fullwidth}

% r.5 contents
\tableofcontents

\listoffigures

\listoftables

%\chapter*{List of Acronyms}
%\acuseall%
\printacronyms[pages={display=all,seq/use=false}]

% r.9 introduction
\cleardoublepage
\chapter*{Introduction}

This sample book discusses the design of Edward Tufte's
books\cite{Tufte2001,Tufte1990,Tufte1997,Tufte2006}.


%%
% Start the main matter (normal chapters)
\mainmatter


\chapter{Reliability Engineering in CubeSat Systems}
\label{sec:fdir}


%\newthought{The pages} of a book are usually divided into three major
%sections: the front matter (also called preliminary matter or prelim), the
%main matter (the core text of the book), and the back matter (or end
%matter).

%\newthought{The front matter} of a book refers to all of the material that
%comes before the main text.  The following table from shows a list of
%material that appears in the front matter of
%along with its page number.  Page numbers that appear in parentheses refer
%to folios that do not have a printed page number (but they are still
%counted in the page number sequence).

\section{Kalispera}
space is very important
\acf{FDIR}

\chapter{The AcubeSAT mission}

\chapter{\ac{FDIR} concept in AcubeSAT}

\chapter{Software implementation of \ac{FDIR}}

\chapter{Hardware implementation of \ac{FDIR}}


\backmatter

\bibliography{sample-handout}
\bibliographystyle{plainnat}


\printindex

\end{document}

